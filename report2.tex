\documentclass{article}

\usepackage[margin=1in]{geometry}
\usepackage{amsmath,amssymb}
\usepackage[table,xcdraw]{xcolor}
\usepackage{graphicx}
\usepackage{caption}
\usepackage{subcaption}

\graphicspath{ {ReportImages/OneClassSVM/} }


\begin{document}

\title{Autonomous Agents\\
Assignment 2: \emph{Single Agent Learning}}
\author{
Artur Alkaim -- 10859368\\
Peter Dekker -- 10820973\\
Rafael Reia -- 10859454\\
Yikang Wang -- 10540288\\
}
\maketitle
\section{Introduction}
In this assignment, we study the application of \emph{reinforced learning
algorithms}. The main focus of this study is on the Q-Learning algorithm where
we used a \epsilon-policy at the begining and changed other policies to experiment
the efects of that.

Our goal is to tune the algorithms with the different parameters and understand
how this parameters affect the performance that in this specific instance is
determined by the number of steps the predator needs to catch the prey and how
we can improve that with learning. 

\section{Program design}
In this assignment the overall progam design stays the same, as we just added
the equivalent classes for the new algorithms. So we have a new main class,
\emph{MainQl} that, like the others, create a specific environment for the
predator and prey to live. Then it runs the simulation for that setup.

We also added some utility classes to produce the graphs for this report.

\section{Evaluation}

%- Why? 'In order to test\ldots'
%- How? ' we run \ldots N times on X computer'
%- What does it show?
%- Take home message

In order to test how the Q-learning algorithm works with different parameters we
have run our implemetation and computed the average of $100$ runs, each of them
with $10000$ episodes. With the averaging we have been able to get less sharper
graphs.

\subsection{Different \alpha's}
We started to experiment with different values of \alpha. The values used are:
$0.1 , 0.2, 0.3, 0.4 and 0.5$. 

\subsection{Different \gamma's}
We started to experiment with different values of \gamma. The values used are:
$0.2, 0.5, 0.7 and 0.9$. 



\section{Conclusion}


\end{document}
